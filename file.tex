\documentclass[12pt]{article}
\usepackage[utf8]{inputenc}
\usepackage{setspace}
\usepackage{geometry}
\geometry{a4paper, margin=1in}

% Title Page Information
\title{\textbf{Cognitive and Neural Mechanisms of Learning: An Exploration of Basal Ganglia and Dopamine}}
\author{Rakshith Vaishnavi Dogra\\Roll No: 21218 \\Department: Data Science and Engineering \\Indian Institute of Science Education and Research (IISER), Bhopal}
\date{}

\begin{document}

\maketitle

\begin{abstract}
\noindent This reflection paper reviews the key insights from Shohamy et al.'s study on the role of the basal ganglia and dopamine in probabilistic category learning. The study highlights how distinct memory systems, such as the basal ganglia and medial temporal lobes, contribute to different aspects of learning, particularly focusing on incremental, feedback-based learning mechanisms. By analyzing neuroimaging, neuropsychological, and computational studies, the authors identify how feedback and dopamine modulate learning processes. The reflection explores the relevance of these findings in real-life decision-making, particularly in conditions like Parkinson's disease, and discusses the broader implications of understanding cognitive learning systems.
\end{abstract}

\newpage

\onehalfspacing

\section*{Summary}
This paper investigates the contributions of the basal ganglia and dopamine in learning, with a specific focus on probabilistic category learning. The basal ganglia are shown to play a crucial role in feedback-based, incremental learning by integrating information over multiple experiences. In contrast, the medial temporal lobe (MTL) supports the rapid encoding of relational memories. 

A significant part of the study involves the "weather prediction" task, where participants predict outcomes based on probabilistic cues. This task highlights how the basal ganglia contribute to incremental learning while the MTL supports episodic memory. Studies with Parkinson's patients reveal that damage to the basal ganglia leads to impairments in feedback-based learning but not in observational learning, emphasizing the specificity of the basal ganglia in processing feedback.

Dopamine's role in learning is particularly emphasized in relation to reward prediction errors. Dopamine neurons signal whether outcomes are better or worse than expected, thereby adjusting behavior. Computational models reveal that dopamine modulates error correction, allowing for the optimization of learning strategies over time. The study also demonstrates that Parkinson's patients struggle with integrating feedback over time but perform adequately in tasks relying on single-cue associations.

The authors further explore how disruptions in these neural systems impact learning patterns. The results suggest that while the basal ganglia handle feedback-based stimulus-response learning, the MTL facilitates flexible generalization. This distinction is crucial for designing therapeutic interventions for neurological disorders.

\section*{Reflection}
The findings from the paper's study on probabilistic category learning have significant real-world implications, especially in understanding how individuals make decisions under uncertainty. The distinction between feedback-based and observational learning is particularly relevant in designing interventions for conditions like Parkinson's disease. By leveraging the spared capabilities in patients, educational and therapeutic strategies can be tailored to enhance learning outcomes.

The study's exploration of dopamine's role in reward prediction has changed my perspective on how humans adapt to changing environments. Previously, I viewed learning as a process focused on memory retention. However, this paper highlights the dynamic nature of learning, where feedback plays a critical role in shaping decision-making strategies.

The study raises questions about the generalization of these findings to more complex scenarios. Can the same principles apply to ambiguous, real-world decision-making processes where feedback is not always immediate or clear? Developing computational models to predict learning patterns in such settings could enhance our understanding of personalized learning strategies.

Lastly, I agree with the authors’ conclusion that different neural systems interact to support learning. This understanding can inform cognitive rehabilitation programs by 
tailoring interventions to leverage intact neural pathways.


\section*{Bibliography}
\noindent Shohamy, D., Myers, C. E., Kalanithi, J., \& Gluck, M. A. (2008). Basal ganglia and dopamine contributions to probabilistic category learning. \textit{Neuroscience \& Biobehavioral Reviews, 32}(2), 219-236. \texttt{doi:10.1016/j.neubiorev.2007.07.008}.

\noindent Aron, A. R., Shohamy, D., Clark, J., Myers, C., Gluck, M. A., \& Poldrack, R. A. (2004). Human midbrain sensitivity to cognitive feedback and uncertainty during classification learning. \textit{Journal of Neurophysiology, 92}(2), 1144–1152. 
\texttt{https://doi.org/10.1152/jn.01209.2003}
\end{document}
